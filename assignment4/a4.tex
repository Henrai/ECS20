% A useful template for typesetting beautiful homework solutions. 
% Also check out Professor Matloff's guide: 
% http://heather.cs.ucdavis.edu/~matloff/LaTeX/HowToCreate.html.
\documentclass{article}

% Packages Used
\usepackage{fancyhdr} % Required for custom headers
\usepackage{lastpage} % Required to determine the last page for the footer
\usepackage{extramarks} % Required for headers and footers
\usepackage{graphicx} % Required to insert images
\usepackage{lipsum} % Used for inserting dummy 'Lorem ipsum' text into the template
\usepackage{comment}  % Used for multi-line commenting
\usepackage{booktabs} % For better looking tables
\usepackage{array}       % for better arrays (eg matrices) in maths
\usepackage{paralist}    % very flexible & customisable lists (eg. enumerate/itemize, etc.)
\usepackage{verbatim}  % adds environment for commenting out blocks of text & for better verbatim
\usepackage{subfig}      % make it possible to include more than one captioned figure/table in a single float
\usepackage{amsthm}   % make proofs look better
\usepackage{amsfonts}
\usepackage{amsmath}
\usepackage{amssymb}
\usepackage{eufrak}      % for fraktur fonts
\usepackage{mathabx}  % for \divides
\usepackage{enumerate} % to get lists enumerated with letters
\usepackage{hyperref}  % to get attractive URLs
\usepackage{bussproofs} % for setting proofs
\usepackage{etoolbox}
\usepackage{enumitem}
\usepackage{algorithm}
\usepackage{algorithmic}
\usepackage{tikz}
% For theorem enviornment
\theoremstyle{definition}
\newtheorem{definition}{Definition}
\newtheorem{theorem}{Theorem}[section]
\newtheorem{corollary}{Corollary}[theorem]
\newtheorem{lemma}{Lemma}

\newtheorem{mathrule}{Rule}
\newtheorem{case}{Case}
\newtheorem{subcase}{Case}[case]

\theoremstyle{plain}
\newtheorem{example}{Example}
\newtheorem{problem}{Problem}[section]

% For improved end of proof formatting
\patchcmd{\endproof}  % <cmd>
  {\endtrivlist}               % <search>
  {\endtrivlist\par\nobreak\vspace*{\dimexpr-\baselineskip-\parskip}\nobreak\noindent\hrulefill}% <replace>
  {}{}                            % <succes><failure>

% Margins
\topmargin=-0.45in
\evensidemargin=0in
\oddsidemargin=0in
\textwidth=6.5in
\textheight=9.0in
\headsep=0.25in 

\linespread{1.1} % Line spacing

% Set up the header and footer
\pagestyle{fancy}
\lhead{ECS20: Discrete Mathematics\\ UC Davis - Patrice Koehl} % Top left header
\chead{} % Top center header
\rhead{\firstxmark Anze Wang ID: 912777492\\ECS 020 A03} % Top right header
\lfoot{\lastxmark} % Bottom left footer
\cfoot{} % Bottom center footer
\rfoot{Page\ \thepage\ of\ \pageref{LastPage}} % Bottom right footer

\setlength\parindent{10pt} % Removes all indentation from paragraphs

% Common boolean operators.
\newcommand*\AND{\wedge}
\newcommand*\OR{\vee}
\newcommand*\NOT{\neg}
\newcommand*\IMPLIES{\implies}
\newcommand*\XOR{\mathbin{\oplus}}


\begin{document}

\begin{center} \bf \LARGE Homework 4\\
\end{center}


\begin {enumerate}[itemindent=30pt,label=\bf Exercise {\arabic*}:]

\item .\\
Give a direct proof, an indirect proof, and a proof by contradiction of the statement: if n is even, then n+4 is even.
\subitem p: n is even; q: n+4 is even\\
\subitem direct proof: $p \to q$
\subitem Because n is even, let n = 2k. then n + 4 = 2k +4, which is even.\\
\subitem indirect proof: $\neg q \to \neg p$
\subitem Because n + 4 is odd, n+4 = 2k +1. Then n = 2k - 3, which is odd.\\
\subitem proof by contradiction: $\neg p \to q$ 
\subitem Because n is odd, n = 2k + 1. Then n + 4 = 2k + 5, which is odd. This is contradict with the statement n +4 is even. So if n is even, n+4 is even.
\item .\\
Let A, B and C be sets. Show that $(A-B)-C = (A-C)-(B-C)$
\begin{align*}
	&\;\;\;\;(A-C)-(B-C) \\
	&=\{x|(x \in A \land x \notin C) \land \neg (x \in B \land x \notin C) \}\\
	&=\{x|(x \in A \land x \notin C) \land (x \notin B \lor x \in C) \}\\
	&=\{x|x \in A \land (x \notin C \land (x \notin B \lor x \in C)) \}\\	
	&=\{x|x \in A \land ((x \notin C \land x \notin B) \lor (x \notin C \land x \in C)) \}\\	
	&=\{x|x \in A \land (x \notin C \land x \notin B)\}\\
	&=\{x|(x \in A \land x \notin B) \land x \notin C\}\\	
	&=(A-B)-C
\end{align*}
\item .\\
Show that $A \oplus B = (A - B) \cup (B - A)$
\subitem According to the definition, we know that
\begin{align*}
	&\;\;\;\;A \oplus B\\
	&=\{x|(x \in A \lor x \in B) \land \neg (x \in A \land x \in B)\}\\
	&=\{x|(x \in A \lor x \in B) \land (x \notin A \lor x \notin B)\}\\
	&=\{x|(x \in A \land (x \notin A \lor x \notin B)) \lor (x \in B \land (x \notin A \lor x \notin B))\}\\
	&=\{x|(x \in A \land x \notin B) \lor (x \in B \land x \notin A )\}\\
	&=(A - B) \cup (B - A)
\end{align*} 
\newpage
\item .
\subitem (a) show that $A \oplus B = B \oplus A$
\begin{align*}
	&\;\;\;\;A \oplus B\\
	&=\{x|(x \in A \lor x \in B) \land \neg (x \in A \land x \in B)\}\\
	&=\{x|(x \in B \lor x \in A) \land \neg (x \in B \land x \in A)\}\\
	&=B \oplus A	
\end{align*}
\subitem (b) show that $(A \oplus B) \oplus B = A$
\begin{align*}
	&\;\;\;\;(A \oplus B) \oplus B\\
	&=\{x|(x \in A \oplus x \in B) \oplus x \in B\}\\
	&=\{x| x \in A \oplus (x \in B \oplus x \in B) \}\\
	&=\{x|x \in A\}\\
	&=A
\end{align*}
\subitem (c) show that $A \neq A \oplus A$ if A is a non empty set.
\begin{align*}
	&\;\;\;\;A \oplus A\\
	&=\{x| x \in A \oplus x \in A\}\\
	&=\varnothing
\end{align*}
\subitem \qquad\qquad\qquad$\therefore$ if A is not empty set, A cannot equal to $A \oplus A$
\item .\\
Can you conclude that A = B if A, B, and C are sets such that:
\subitem (a) $A \cup C = B \cup C $
\subitem \qquad if $A = \{1\}$, $B =\{2\}$, $C=\{1,\;2,\;3\}$
\subitem \qquad then,$A \cup C = B \cup C = \{1,\;2,\;3\} $
\subitem \qquad However, $A \neq B$
\subitem \qquad So I cannot conclude that A = B if $A \cup C = B \cup C $
\subitem (b) $A \cap C = B \cap C $
\subitem \qquad if $A = \{1.\;3,\;4,\;5\}$, $B =\{2,\;3,\;4,\;5\}$, $C=\{3,\;4,\;5\}$
\subitem \qquad then,$A \cap C = B \cap C = \{3,\;4,\;5\} $
\subitem \qquad However, $A \neq B$
\subitem \qquad So I cannot conclude that A = B if $A \cap C = B \cap C $
\newpage
\item .\\
Show that if A, B, and C are sets then
$$|A \cup B \cup C|\;=\;|A|+|B|+|C|-|A \cap B|-|A \cap C|-|B \cap C| + |A \cap B \cap C|$$
\begin{align*}
	&\quad\; |A \cup B \cup C|\\
	&=|A \cup (B \cup C)|\\
	&=|A|+ |B \cup C| - |A \cap (B \cup C)|\\
	&=|A|+ |B \cup C| - |(A \cap B) \cup  (A \cap C)|\\
	&=|A|+ |B \cup C| - |A \cap B|  -  |A \cap C| + |A \cap B \cap C|\\
	&=|A|+|B|+|C|-|A \cap B|-|A \cap C|-|B \cap C| + |A \cap B \cap C|
\end{align*}
%\begin{figure}[h]
%\begin{tikzpicture}
%	\draw ( 90:1.2) circle (2);
%	\draw (210:1.2) circle (2); 
%	\draw (330:1.2) circle (2);
%	\node at ( 90:2) {A}; 
%	\node at (210:2) {B}; 
%	\node at (330:2) {C};
%	\node at (160:1.3) {$A \cap B$};
%	\node at (270:1.2) {$B \cap C$};
%	\node at (380:1.4) {$A \cap C$};
%	\node at (260: 0.2) {$A \cap B \cap C$};
%	\node [right, text width = 9cm, align = justify] at (3.5,1.5) {
%          The area of the venn diagram is the sum of A,B and C minus the area of $A \cap B$, $B \cap C$ and $A \cap C$, the add the area of $A \cap B \cap C$. So $|A \cup B \cup C|\;=\;|A|+|B|+|C|-|A \cap B|-|A \cap C|-|B \cap C| + |A \cap B \cap C|$};
%\end{tikzpicture}
%\end{figure}
\item .\\
Let A and B be subsets of the finite universal set U. Show that: $$| \overline{A} \cap \overline{B}| = |U| - |A| -|B| + |A \cap B|$$
\begin{align*}
	&\quad\;| \overline{A} \cap \overline{B}|\\
	&=|\overline{A \cup B}|\\
	&=|U| - |A \cup B|\\
	&=|U| - |A| -|B| + |A \cap B|\\
\end{align*}
%\begin{tikzpicture}
%\filldraw[fill=gray] (-2,-2) rectangle (3,2);
%\scope % A \cap B
%\clip (0,0) circle (1);
%\fill[white] (0,0) circle (1);
%\endscope
%\scope % A \cap B
%\clip (1,0) circle (1);
%\fill[white] (1,0) circle (1);
%\endscope
% outline
%\draw (0,0) circle (1)
%      (1,0) circle (1);
%\node at (-0.5,0) {A};
%\node at (1.5,0) {B};
%\node at (-1,1) {U};
%\node at (0.5, 0) {$A \cap B$};
%\node [right, text width = 9cm, align = justify] at (3.5,1) {
%	The area of $| \overline{A} \cap \overline{B}|$ is the shade area, which is qual to the area of U minus the area of A and B, and add the area of the intersection of A and B. So  $| \overline{A} \cap \overline{B}| = |U| - |A| -|B| + |A \cap B|$
%};
%\end{tikzpicture}
\item .\\
let $A_{i} = \{...,\;-2,\;-1,\;0,\;1,\;2,...,\;i\}$. Find:
\subitem a) $\bigcup \limits_{i=1}^{n} A_{i} = A_{n} = \{...,\;-2,\;-1,\;0,\;1,\;2,...,\;n\}$
\subitem b) $\bigcap \limits_{i=1}^{n} A_{i} = A_{1} = \{...,\;-2,\;-1,\;0,\;1\}$
\item .\\
Let A and B be two sets. Show that if $A \cup B = B$ then $A \cap B = A$
\begin{align*}
	&\because A \cup B = B\\
	&\therefore A \subseteq B\\
	&\therefore A \cap B = A\\
\end{align*}
\newpage
\item .\\
Let A and B be two sets. Show that if $A \cap B = A$ then $B \cap (\overline{B \cap \overline{A}}) = A$
\begin{align*}
	&\quad\; B \cap (\overline{B \cap \overline{A}})\\
	&= B \cap (\overline{B} \cup A)\\
	&= (B \cap \overline{B}) \cup  (B \cap A)\\
	&= \phi \cup (B \cap A)\\
	&= B \cap A\\
	&= A
\end{align*}
\item Extra credit:\\
A 4x4 magic square is an arrangement of the number from 1 to 16 in a 4x4 table, such that the sum of the elements on any rows, columns and main diagonals is the same. Show that the square shown below cannot be completed to be a magic square:
\subitem 
\begin{tabular}{| c | c | c | c |}
\hline
	1&2&3&?\\
\hline
	?&4&5&6\\
\hline 
	7&?&8&?\\
\hline
	?& 9&?&10\\
\hline
\end{tabular}
\subitem  Because $1+4+8+10=23$, we can know that the sum of each line is 23.
\subitem Therefore, The question mark in the first line must be 23 - 1 - 2 -3 = 17, which is bigger than 16.
\subitem So this square cannot be completed.
\end{enumerate}
\end{document}
 

 