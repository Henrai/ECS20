% Also check out Professor Matloff's guide: 
% http://heather.cs.ucdavis.edu/~matloff/LaTeX/HowToCreate.html.
\documentclass{article}

% Packages Used
\usepackage{fancyhdr} % Required for custom headers
\usepackage{lastpage} % Required to determine the last page for the footer
\usepackage{extramarks} % Required for headers and footers
\usepackage{graphicx} % Required to insert images
\usepackage{lipsum} % Used for inserting dummy 'Lorem ipsum' text into the template
\usepackage{comment}  % Used for multi-line commenting
\usepackage{booktabs} % For better looking tables
\usepackage{array}       % for better arrays (eg matrices) in maths
\usepackage{paralist}    % very flexible & customisable lists (eg. enumerate/itemize, etc.)
\usepackage{verbatim}  % adds environment for commenting out blocks of text & for better verbatim
\usepackage{subfig}      % make it possible to include more than one captioned figure/table in a single float
\usepackage{amsthm}   % make proofs look better
\usepackage{amsfonts}
\usepackage{amsmath}
\usepackage{amssymb}
\usepackage{eufrak}      % for fraktur fonts
\usepackage{mathabx}  % for \divides
\usepackage{enumerate} % to get lists enumerated with letters
\usepackage{hyperref}  % to get attractive URLs
\usepackage{bussproofs} % for setting proofs
\usepackage{etoolbox}
\usepackage{enumitem}
\usepackage{tabularx}
%algorithm
\usepackage{algorithmicx}
\usepackage[ruled]{algorithm}
\usepackage{algpseudocode}
\usepackage{algpascal}
\usepackage{algc}
%\algdisablelines
\newcommand{\alg}{\texttt{algorithmicx}}
\newcommand{\old}{\texttt{algorithmic}}
\newcommand{\euk}{Euclid}
\newcommand\ASTART{\bigskip\noindent\begin{minipage}[b]{0.5\linewidth}}
\newcommand\ACONTINUE{\end{minipage}\begin{minipage}[b]{0.5\linewidth}}
\newcommand\AENDSKIP{\end{minipage}\bigskip}
\newcommand\AEND{\end{minipage}}


% For theorem enviornment
\theoremstyle{definition}
\newtheorem{definition}{Definition}
\newtheorem{theorem}{Theorem}[section]
\newtheorem{corollary}{Corollary}[theorem]
\newtheorem{lemma}{Lemma}

\newtheorem{mathrule}{Rule}
\newtheorem{case}{Case}
\newtheorem{subcase}{Case}[case]

\theoremstyle{plain}
\newtheorem{example}{Example}
\newtheorem{problem}{Problem}[section]
\providecommand{\ceil}[1]{\left \lceil #1 \right \rceil }
\providecommand{\floor}[1]{\left \lfloor #1 \right \rfloor }
% For improved end of proof formatting
\patchcmd{\endproof}  % <cmd>
  {\endtrivlist}               % <search>
  {\endtrivlist\par\nobreak\vspace*{\dimexpr-\baselineskip-\parskip}\nobreak\noindent\hrulefill}% <replace>
  {}{}                            % <succes><failure>

% Margins
\topmargin=-0.45in
\evensidemargin=0in
\oddsidemargin=0in
\textwidth=6.5in
\textheight=9.0in
\headsep=0.25in 

\linespread{1.1} % Line spacing

% Set up the header and footer
\pagestyle{fancy}
\lhead{ECS20: Discrete Mathematics\\ UC Davis - Patrice Koehl} % Top left header
\chead{} % Top center header
\rhead{\firstxmark Anze Wang ID: 912777492\\ECS 020 A03} % Top right header
\lfoot{\lastxmark} % Bottom left footer
\cfoot{} % Bottom center footer
\rfoot{Page\ \thepage\ of\ \pageref{LastPage}} % Bottom right footer

\setlength\parindent{10pt} % Removes all indentation from paragraphs

% Common boolean operators.
%\newcommand*\AND{\wedge}
%\newcommand*\OR{\vee}
%\newcommand*\NOT{\neg}
%\newcommand*\IMPLIES{\implies}
%\newcommand*\XOR{\mathbin{\oplus}}


\begin{document}

\begin{center} \bf \LARGE Homework 8\\
\end{center}


\begin {enumerate}[itemindent=30pt,label=\bf Exercise {\arabic*}:]
\item .\\
show that $\sum \limits_{i=1}^{n} i^3 = \bigg(\dfrac{n(n+1)}{2} \bigg)^{2}$ for all $n \geq 1$
\subitem For $n = 1$:
	$$\sum \limits_{i=1}^{1} i^3 = 1 = \bigg(\dfrac{1(1+1)}{2}\bigg)^{2}$$
\subitem So the equation is hold when $n = 1$
\subitem If the equation is held for $n = k,\;k\geq 1$, then
\begin{align*}
	\sum \limits_{i=1}^{k} i^3 &= \bigg(\dfrac{k(k+1)}{2} \bigg)^{2} \\
	\sum \limits_{i=1}^{k+1} i^3 &= \bigg(\dfrac{k(k+1)}{2} \bigg)^{2} + (k+1)^3	\\
	& =\frac{(k+1)^2}{4}(k^2 + 4k + 4) \\
	& =\frac{(k+1)^2}{4}(k+2)^2\\
	&=\bigg(\dfrac{(k+2)(k+1)}{2} \bigg)^{2}
\end{align*}
\subitem Therefore the statement holds when $n=(k + 1)$.
\subitem Since both the basis and the inductive step have been performed, by mathematical induction, the statement holds for all $n \geq 1$, $n \in \mathbb{R}$
 \item .\\
 Use mathematical induction to prove that:
 $$1*2*3+2*3*4+...+n(n+1)(n+2) = \dfrac{n(n+1)(n+2)(n+3)}{4}, \text{for all } n \geq 1$$
\subitem let $f(n) = 1*2*3+2*3*4+...+n(n+1)(n+2)$
\subitem for $n = 1$
\begin{align*}
	f(1) &= 1*2*3 \\
		 &= 6 \\
		 &= \dfrac{1(1+1)(1+2)(1+3)}{4} 
\end{align*}
\subitem So the $f(n)$ holds when $n = 1$
\subitem If the $f(n)$ held when $n = k, k \geq 1$, then
\begin{align*}
	f(k+1) &= f(k) + (k+1)(k+2)(k+3)\\
		   &= \dfrac{k(k+1)(k+2)(k+3)}{4} + (k+1)(k+2)(k+3)\\
		   &= (k+1)(k+2)(k+3)(\dfrac{k}{4} + 1)\\
		   &= \dfrac{(k+1)(k+2)(k+3)(k+4)}{4}		   
\end{align*}
\subitem Therefore $f(n)$ holds when $n=(k + 1)$.
\subitem Since both the basis and the inductive step have been performed, by mathematical induction, the $f(x)$ holds for all $n \geq 1$, $n \in \mathbb{N}$
\item .\\
Prove that $$1 + \dfrac{1}{4} + \dfrac{1}{9} + \ldots + \dfrac{1}{n^2} < 2 - \dfrac{1}{n}$$
Whenever $n$ is a positive integer greater than $1$
\subitem for $n = 2$ $$1 + \dfrac{1}{4} < 2 - \dfrac{1}{2}$$
\subitem so the inequality holds when $n = 2$
\subitem if the inequality holds when $n = k, k \geq 2$, then
$$1 + \dfrac{1}{4} + \dfrac{1}{9} + \ldots + \dfrac{1}{k^2} < 2 - \dfrac{1}{k}$$
\subitem so we can get that $$1 + \dfrac{1}{4} + \dfrac{1}{9} + \ldots + \dfrac{1}{k^2} + \dfrac{1}{(k+1)^2} < 2 - \dfrac{1}{k} + \dfrac{1}{(k+1)^2}$$
\begin{align*}
	&2 - \dfrac{1}{k+1} - \bigg (2 - \dfrac{1}{k} + \dfrac{1}{(k+1)^2}\bigg)\\
   =&\dfrac{k(k+1)(2k+1) - ((2k-1)(k+1)^2 + k)}{k(k + 1)^2}\\
   =&\dfrac{2k^3 + 3k^2 + k - (2k^3 + 3k^2 -2k + 1)}{k(k + 1)^2}\\
   =&\dfrac{2k - 1}{k(k + 1)^2}
\end{align*}
\subitem Because $\dfrac{2k - 1}{k(k + 1)^2 } > 0 \text{ for all }k > 1$, therefore $2 - \dfrac{1}{k}$ is greater than $2 - \dfrac{1}{k} + \dfrac{1}{(k+1)^2}$ for all $k > 1$, thus $1 + \dfrac{1}{4} + \dfrac{1}{9} + \ldots + \dfrac{1}{k^2} + \dfrac{1}{(k+1)^2} < 2 - \dfrac{1}{k+1}$
\subitem Therefore inequality holds when $n=(k + 1)$.
\subitem Since both the basis and the inductive step have been performed, by mathematical induction, the $f(n)$ holds for all $n \geq 1$, $n \in \mathbb{N}$
\item . \\Use mathematical induction to show that $n^2-7n+12$ is non negative if $n$ is an integer greater than $3$.
\subitem let $f(n) = n^2-7n+12$
\subitem for $n = 4$ $f(4) = 4*4 - 7 * 4 + 12 = 0$
\subitem so $f(n)$ holds when $n =4$
\subitem if $f(n)$ holds when $n = k, k > 4$,
\begin{align*}
	f(k + 1) &= (k+1)^2 - 7(k+1) +12 \\
			 &= k^2 +2k +1 - 7k - 7 + 12 \\
			 &= k^2 - 7k + 12 + 2k  + 5\\
			 & = f(k) + 2k + 5
\end{align*}
\subitem because $f(k) > 0$ and $2k + 5 > 0$, $f(k+1) > 0$
\subitem Since both the basis and the inductive step have been performed, by mathematical induction, the $f(n)$ holds for all $n > 3$, $n \in \mathbb{N}$
\item .\\
Use mathematical induction to prove that a set with $n$ elements has $n(n-1)/2$ sub sets containing exactly two elements whenever $n$ is an integer greater than or equal to 2.
\subitem for $n = 2$, it only has one subset. And $2(2-1)/2 = 1$. So this statement holds when $n = 2$
\subitem for $n > 2$, if we add one more element, the new element can form a set with all of the element in the old set. So the total number of sub set will increase k.
$$ \dfrac{k(k-1)}{2} + k = \dfrac{k(k+1)}{2}$$
\subitem Therefore inequality holds when $n=(k + 1)$.
\subitem Since both the basis and the inductive step have been performed, by mathematical induction, the statement holds for all $n \geq 2$, $n \in \mathbb{N}$ 
\item .\\
Find the flaw with the following proof that $a^n=1$ for all non negative integer $n$, when ever $a$ is a non zero real number: Basis step: $a^0=1$ is true, by definition of $a^0$ Inductive step: assume that $a^j=1$ for all non negative integers $j$ with $j \leq k$. Then note that:$$a^{k+1} = \dfrac{a^k \cdot a^k}{a^{k-1}}$$
\subitem because he prove $a^{k+1}$ by using $a^k$ and $a^{k-1}$, he need two basic case to let this induction available. However, he only give $a^0 = 1$. That is why this proof is wrong.

\item .\\
Use mathematical induction to show that 21 divides $4^{n+1} + 5^{2n-1}$ whenever $n$ is a positive integer.
\subitem for $n = 1$, we can get that $4^{n+1} + 5^{2n-1} = 4^2 + 5^1 = 21$
\subitem Therefore this statement holds for base statement.
\subitem if this statement holds for $n = k$, then we assume that $4^{n+1} \;mod\; 21 = i$, and $5^{2n-1}\;mod\; 21 = j$.
\begin{align*}
	&(4^{(n+1) + 1} + 5^{2(n+1) -1}) \;mod\; 21\\
	=&(4*4^{n+1} + 25*5^{2n-1}) \;mod\; 21\\
	=& 4\;mod\;21 * i + 25\;mod\;21*j\\
	=& 4(i + j) \;mod\; 21\\
	=& 0
\end{align*}
\subitem thus this statement holds when $n = k+1$
\subitem Since both the basis and the inductive step have been performed, by mathematical induction, the statement holds for all positive integer.
\item .\\prove that $f_{1}^{2} + f_{2}^{2} + \ldots + f_{n}^{2} = f_{n}f_{n+1}$, whenever n is a positive integer.
\subitem for $n = 1$, $f_1^2 = f_1 f_2 = 1*1 = 1$, so this statement holds when n = 1
\subitem if this statement holds when $n = k$, $k > 1$
\begin{align*}
	&f_{1}^{2} + f_{2}^{2} + \ldots + f_{n}^{2} + f_{n+1}^2\\
	=&f_n f_{n+1} + f_{n+1}^2\\
	= &f_{n+1}(f_{n}+ f_{n+1})\\
	= &f_{n+1}f_{n+2}
\end{align*}
\subitem thus this statement holds when $n = k+1$
\subitem Since both the basis and the inductive step have been performed, by mathematical induction, the statement holds for all positive integer.
\item .\\
show that $f_{0} - f_{1} + f_{2} - \ldots - f_{2n-1} + f_{2n} = f_{2n-1} -1$
\subitem for $n = 1$
\begin{align*}
	LHS &= f_{0} - f_{1} + f_{2}\\
		&= 0 - 1 + 1\\
		&=0\\
	RHS &= f_{1} -1\\
		&= 0\\
\end{align*}
\subitem So this statement holds for $n = 1$
\subitem if this statement holds when $n = k$, $k > 1$
\begin{align*}
	&f_{0} - f_{1} + f_{2} - \ldots - f_{2n-1} + f_{2n} - f_{2n+1} + f_{2n + 2} \\
	=&f_{2n-1}-1 - f_{2n+1} + f_{2n + 2}\\
	=& f_{2n-1}+ f_{2n}) - 1\\
	=&f_{2n+1} -1
\end{align*}
\subitem thus this statement holds when $n = k+1$
\subitem Since both the basis and the inductive step have been performed, by mathematical induction, the statement holds for all positive integer.

\item .\\
Use mathematical induction to prove that a set with $n$ elements has $n(n-1)(n-2)/6$ subsets containing exactly three elements whenever $n$ is an integer greater than or equal to 3.
\subitem for $x = 3$, the number of subset equals to $n(n-1)(n-2)/6 = 3*2*1/6 = 1$
\subitem if this statement holds when $n = k$, $k > 3$
\subitem if we add one more element, the new element can form a set with all of the subset with 2 elements in the old set.As I proved in exercise 5, we can know that a set with $n$ elements has $n(n-1)/2$ sub sets containing exactly two elements whenever $n$ is an integer greater than or equal to 2. So the total number of sub set will increase $k(k-1)/2$.  
\begin{align*}
	&k(k-1)(k-2)/6 +  k(k-1)/2 \\
	=&k(k-1)((k-2)/6 + 1/2) \\
	=&k(k-1)(k+1)/6
\end{align*}
\subitem thus this statement holds when $n = k+1$
\subitem Since both the basis and the inductive step have been performed, by mathematical induction, the statement holds for all positive integer,which is greater than or equal to 3.
\end{enumerate}
\end{document}