% A useful template for typesetting beautiful homework solutions. 
% Also check out Professor Matloff's guide: 
% http://heather.cs.ucdavis.edu/~matloff/LaTeX/HowToCreate.html.
\documentclass{article}

% Packages Used
\usepackage{fancyhdr} % Required for custom headers
\usepackage{lastpage} % Required to determine the last page for the footer
\usepackage{extramarks} % Required for headers and footers
\usepackage{graphicx} % Required to insert images
\usepackage{lipsum} % Used for inserting dummy 'Lorem ipsum' text into the template
\usepackage{comment}  % Used for multi-line commenting
\usepackage{booktabs} % For better looking tables
\usepackage{array}       % for better arrays (eg matrices) in maths
\usepackage{paralist}    % very flexible & customisable lists (eg. enumerate/itemize, etc.)
\usepackage{verbatim}  % adds environment for commenting out blocks of text & for better verbatim
\usepackage{subfig}      % make it possible to include more than one captioned figure/table in a single float
\usepackage{amsthm}   % make proofs look better
\usepackage{amsfonts}
\usepackage{amsmath}
\usepackage{amssymb}
\usepackage{eufrak}      % for fraktur fonts
\usepackage{mathabx}  % for \divides
\usepackage{enumerate} % to get lists enumerated with letters
\usepackage{hyperref}  % to get attractive URLs
\usepackage{bussproofs} % for setting proofs
\usepackage{etoolbox}
\usepackage{enumitem}
\usepackage{algorithm}
\usepackage{algorithmic}

% For theorem enviornment
\theoremstyle{definition}
\newtheorem{definition}{Definition}
\newtheorem{theorem}{Theorem}[section]
\newtheorem{corollary}{Corollary}[theorem]
\newtheorem{lemma}{Lemma}

\newtheorem{mathrule}{Rule}
\newtheorem{case}{Case}
\newtheorem{subcase}{Case}[case]

\theoremstyle{plain}
\newtheorem{example}{Example}
\newtheorem{problem}{Problem}[section]

% For improved end of proof formatting
\patchcmd{\endproof}  % <cmd>
  {\endtrivlist}               % <search>
  {\endtrivlist\par\nobreak\vspace*{\dimexpr-\baselineskip-\parskip}\nobreak\noindent\hrulefill}% <replace>
  {}{}                            % <succes><failure>

% Margins
\topmargin=-0.45in
\evensidemargin=0in
\oddsidemargin=0in
\textwidth=6.5in
\textheight=9.0in
\headsep=0.25in 

\linespread{1.1} % Line spacing

% Set up the header and footer
\pagestyle{fancy}
\lhead{ECS20: Discrete Mathematics\\ UC Davis - Patrice Koehl} % Top left header
\chead{} % Top center header
\rhead{\firstxmark Anze Wang ID: 912777492\\ECS 020 A03} % Top right header
\lfoot{\lastxmark} % Bottom left footer
\cfoot{} % Bottom center footer
\rfoot{Page\ \thepage\ of\ \pageref{LastPage}} % Bottom right footer

\setlength\parindent{10pt} % Removes all indentation from paragraphs

% Common boolean operators.
\newcommand*\AND{\wedge}
\newcommand*\OR{\vee}
\newcommand*\NOT{\neg}
\newcommand*\IMPLIES{\implies}
\newcommand*\XOR{\mathbin{\oplus}}


\begin{document}

\begin{center} \bf \LARGE Homework 3\\
\end{center}


\begin {enumerate}[itemindent=30pt,label=\bf Exercise {\arabic*}:]

\item .\\
Show that this implication is a tautology, by using a truth table:
\subitem \qquad $[(p \lor q) \land (p \to r) \land (q \to r)] \to r$
\subitem 
\begin{tabular}{| c | c | c | c | c | c | c |}
	\hline 
	p & q & r & $p \lor q$ & $p \to r$ & $ q \to r$ & $[(p \lor q) \land (p \to r) \land (q \to r)] \to r$\\
	\hline
	T & T & T & T & T & T & T\\
	\hline	
	T & T & F & T & F & F & T\\
	\hline 
	T & F & T & T & T & T & T \\
	\hline	
	T & F & F & T & F & T & T\\
	\hline 
	F & T & T & T & T & T & T\\
	\hline	
	F & T & F & T & T & F & T\\
	\hline
	F & F & T & F & T & T & T\\
	\hline
	F & F & F & F & T & T & T\\
	\hline
\end{tabular}
\item .\\
Show that $(p \lor q) \land (\neg p \lor r) \to (q \lor r)$ is a tautology.
\subitem 
\begin{tabular}{| c | c | c | c | c | c | c |}
	\hline 
	p & q & r & $p \lor q$ & $\neg p \lor r$ & $ q \lor r$ & $(p \lor q) \land (\neg p \lor r) \to (q \lor r)$\\
	\hline
	T & T & T & T & T & T & T\\
	\hline	
	T & T & F & T & F & F & T\\
	\hline 
	T & F & T & T & T & T & T \\
	\hline	
	T & F & F & T & F & F & T\\
	\hline 
	F & T & T & T & T & T & T\\
	\hline	
	F & T & F & T & F & T & T\\
	\hline
	F & F & T & F & T & T & T\\
	\hline
	F & F & F & F & T & F & T\\
	\hline
\end{tabular}
\item .\\
Determine whether these are valid arguments:
\paragraph{(a)}
``If $x^{2}$ is irrational, then x is irrational. Therefore, if x is irrational, it follows that $x^{2}$is irrational.``
\subitem if $x = \sqrt{2}, then x^{2} = 2$
\subitem $\because \sqrt{2}$ is irrational, and 2 is rational
\subitem $\therefore$ this statement is not valid.
\paragraph{(b)}
``if $x^{2}$ is irrational, then x is irrational. The number $y = \pi^{2}$ is irrational. Therefore, the number $x = \pi$ is irrational.
\subitem $\forall x \quad x^{2} \in$ irrational number,$x \in$ irrational number
\subitem $\because \pi^{2} \in$ irrational number
\subitem $\therefore \pi \in$ irrational number
\newpage
\item .\\Prove that a square of an integer ends with a 0, 1, 4, 5 6 or 9. (Hint:let $n = 10 k+\ell$, where $\ell$ = 0, 1, ...,9)
\subitem let $n = 10 k + \ell, n \in \mathbb{Z}, k \in \mathbb{Z}$
\subitem $n^{2} = 100 k^{2} + 20 k \ell + \ell^{2}$
\subitem $\therefore n^{2}\;mod\;10 = \ell^{2}\;mod\;10 $
\\
\subitem
\begin{tabular}{| c | c | c | c |}
	\hline
	$\ell$ & $\ell^{2}\;mod\;10$ & l & $\ell^{2}\;mod\;10$ \\
	\hline 
	0 & 0 & 5 & 5 \\
	\hline	
	1 & 1 & 6 & 6 \\
	\hline
	2 & 4 & 7 & 9 \\
	\hline 
	3 & 9 & 8 & 4 \\
	\hline
	4 & 6 & 9 & 1 \\
	\hline	
\end{tabular}
\subitem according to the table above, we can know that a square of an integer only can ends with a 0, 1, 4, 5 6 or 9.
\item.\\
Prove that if n is a positive integer, then n is even if and only if 7n+4 is even.
\subitem p : "n is even" q : "7n+4 is even"
\subitem we want to proof that $p \leftrightarrow q$
\paragraph{1)}
$p \to q$
\subitem if n is even, we can assume that $n = 2k, k \in \mathbb{Z}$
\subitem $\therefore 7n + 4 = 14k + 4$ 
\subitem $\because 14k + 4 mod 2 \equiv 0$
\subitem $\therefore q$ is true
\subitem $p \to q$ is true
\paragraph{2)}
$q \to p$
\subitem we can proof it by contrapositive, so we should prove that $\neg p \to \neg q$ is true.
\subitem if n is odd, we can assume that $n = 2k + 1, k \in \mathbb{Z}$
\subitem $\therefore 7n + 4 = 14k + 11$ 
\subitem $\because 14k + 4 mod 2 \equiv 1$
\subitem $\therefore \neg q$ is true
\subitem $\therefore \neg p \to \neg q$ is true
\subitem $\therefore q \to p$ is true
\subitem in the case that $(q \to p) \land (p \to q)$ is true, we can conclude that $p \leftrightarrow q$ is true
\subitem so n is even if and only if 7n+4 is even.
\newpage

\item .\\
Prove that either $2.10^{500} + 15$ or $2.10^{500}+ 16$ is not a perfect square. Is your proof constructive, or non-constructive?
\subitem first of all, I want to prove that the difference between two perfect square cannot be 1.
\subitem suppose we have two integers, m and n $(m < n)$. 
\subitem $n^{2} - m^{2} = (n-m)(n+m)$
\subitem $\because n-m \geqslant 1, n + m > 1$
\subitem $\therefore n^{2} - m^{2} > 1$
\subitem so the difference between two perfect square cannot be 1.
\subitem $\because 2.10^{500} + 16 - 2.10^{500}+ 15 = 1$
\subitem $\therefore$ there are at least 1 number is not perfect sauare number
\subitem so we can conclude that either $2.10^{500} + 15$ or $2.10^{500}+ 16$ is not a perfect square.
\subitem because I do not proof that either $2.10^{500} + 15$ or $2.10^{500}+ 16$ is not a perfect square directly, this proof is constructive. 
\\
\item .\\
Prove or disprove that if a and b are rational numbers, then $a^{b}$ is also rational.
\subitem if a = 2 and b = 0.5.  The number, $2^{0.5}$ will be a irrational number. So this statement is false. 
\item .\\
Prove that at least one of the real numbers $a_{1}, a_{2}, ..., a_{n}$ is greater than or equal to the average of these numbers. What kind of proof did you use?
\subitem if all of the number is less than the average, we can say that $a_{i} = \bar{a} - k_{i},\;k_{i} \in \mathbb{Z}^{+}$
\subitem $\bar{a} = \dfrac{1}{n} \sum\limits_{i=1}^n \bar{a} - k_{i} = \bar{a} - \sum\limits_{i=1}^n k_{i} $
\subitem $\because k_{i} > 0$
\subitem $\therefore$ we get that $\bar{a} < \bar{a}$, which is imposable.
\subitem So my hypothesis is false, which means that there are at least one of the real numbers $a_{1}, a_{2}, ..., a_{n}$ is greater than or equal to the average of these numbers.
\subitem I proof is by contrapositive.
\newpage
\item .\\
The proof below has been scrambled. Please put it back in the correct order.\\
\textbf{claim}: For all $n \geqslant 9$, if n is a perfect square, then n-1 is not prime.\\
1: Since (n-1) is the product of 2 integers greater than 1, we know (n-1) is not prime\\
2: Since $m \geqslant 3$, it follows that $m-1 \geqslant 2$ and $m+1 \geqslant 4$\\
3: Let n be a perfect square such that $n \geqslant 9$ \\
4: This means that $n-1 = m^{2}-1 = (m-1)(m+1)$\\
5: There is an integer $m \geqslant 3$ such that $n = m^{2}$\\
\subitem Answer:
\subitem $3 \to 5 \to 4 \to 2 \to 1$
\item .\\
Prove that these four statements are equivalent: (i) $n^{2}$is odd, (ii) 1-n is even, (iii) $n^{3}$ is odd, (iv) $n^{2}+1$ is even.
\subitem if n is odd, we can assume that $n = 2k + 1$
\subitem $n^{2} = 4k^{2} + 4k +1$
\subitem $\therefore n^{2}$ is odd
\subitem $1 - n = -2k$
\subitem $\therefore 1 - n$ is even
\subitem $n^{3} = 8k^{3} + 4k^{2} + 2k + 1$
\subitem $\therefore n^{3}$ is odd
\subitem $n^{2} + 1 = 4k^{2} + 4k + 2$
\subitem $\therefore n^{2} + 1$ is even 
\subitem if n is even, we can assume that $n = 2k$
\subitem $n^{2} = 4k^{2}$
\subitem $\therefore n^{2}$ is even
\subitem $1 - n = 1-2k$
\subitem $\therefore 1 - n$ is odd
\subitem $n^{3} = 8k^{3}$
\subitem $\therefore n^{3}$ is even
\subitem $n^{2} + 1 = 4k^{2} + 1$
\subitem $\therefore n^{2} + 1$ is odd
\subitem when n is odd, all of the statements are true. When n is even, all of the statements are false. So all of the statements are equivalent.

\item .\\
Use Exercise 8 to show that if the first 10 strictly positive integers are placed around a circle, in any order, then there exist three integers in consecutive locations around the circle that have a sum greater than or equal to 17.
\subitem let $A_{1} = 1 + 2 + 3,\;A_{2} =  2 + 3 + 4\;...\;A_{10} = 10 + 1 + 2$ 
\subitem $\therefore \sum\limits_{i=1}^{10} A_{i} = 55*3 = 165$
\subitem $\therefore$ the average number of $A_{i}$ is 16.5.
\subitem $\because$ all of the numbers are integers 
\subitem $\therefore$ there must have a sum which is greater than or equal to 17.
\end{enumerate}
\end{document}









