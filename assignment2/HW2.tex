% A useful template for typesetting beautiful homework solutions. 
% Also check out Professor Matloff's guide: 
% http://heather.cs.ucdavis.edu/~matloff/LaTeX/HowToCreate.html.
\documentclass{article}

% Packages Used
\usepackage{fancyhdr} % Required for custom headers
\usepackage{lastpage} % Required to determine the last page for the footer
\usepackage{extramarks} % Required for headers and footers
\usepackage{graphicx} % Required to insert images
\usepackage{lipsum} % Used for inserting dummy 'Lorem ipsum' text into the template
\usepackage{comment}  % Used for multi-line commenting
\usepackage{booktabs} % For better looking tables
\usepackage{array}       % for better arrays (eg matrices) in maths
\usepackage{paralist}    % very flexible & customisable lists (eg. enumerate/itemize, etc.)
\usepackage{verbatim}  % adds environment for commenting out blocks of text & for better verbatim
\usepackage{subfig}      % make it possible to include more than one captioned figure/table in a single float
\usepackage{amsthm}   % make proofs look better
\usepackage{amsfonts}
\usepackage{amsmath}
\usepackage{amssymb}
\usepackage{eufrak}      % for fraktur fonts
\usepackage{mathabx}  % for \divides
\usepackage{enumerate} % to get lists enumerated with letters
\usepackage{hyperref}  % to get attractive URLs
\usepackage{bussproofs} % for setting proofs
\usepackage{etoolbox}
\usepackage{enumitem}

% For theorem enviornment
\theoremstyle{definition}
\newtheorem{definition}{Definition}
\newtheorem{theorem}{Theorem}[section]
\newtheorem{corollary}{Corollary}[theorem]
\newtheorem{lemma}{Lemma}

\newtheorem{mathrule}{Rule}
\newtheorem{case}{Case}
\newtheorem{subcase}{Case}[case]

\theoremstyle{plain}
\newtheorem{example}{Example}
\newtheorem{problem}{Problem}[section]

% For improved end of proof formatting
\patchcmd{\endproof}  % <cmd>
  {\endtrivlist}               % <search>
  {\endtrivlist\par\nobreak\vspace*{\dimexpr-\baselineskip-\parskip}\nobreak\noindent\hrulefill}% <replace>
  {}{}                            % <succes><failure>

% Margins
\topmargin=-0.45in
\evensidemargin=0in
\oddsidemargin=0in
\textwidth=6.5in
\textheight=9.0in
\headsep=0.25in 

\linespread{1.1} % Line spacing

% Set up the header and footer
\pagestyle{fancy}
\lhead{ECS20: Discrete Mathematics\\ UC Davis - Patrice Koehl} % Top left header
\chead{} % Top center header
\rhead{\firstxmark Anze Wang ID: 912777492\\ECS 020 A03} % Top right header
\lfoot{\lastxmark} % Bottom left footer
\cfoot{} % Bottom center footer
\rfoot{Page\ \thepage\ of\ \pageref{LastPage}} % Bottom right footer

\setlength\parindent{10pt} % Removes all indentation from paragraphs

% Common boolean operators.
\newcommand*\AND{\wedge}
\newcommand*\OR{\vee}
\newcommand*\NOT{\neg}
\newcommand*\IMPLIES{\implies}
\newcommand*\XOR{\mathbin{\oplus}}


\begin{document}

\begin{center} \bf \LARGE Homework two\\
\end{center}


\begin {enumerate}[itemindent=30pt,label=\bf Exercise {\arabic*}:]

\item .\\
Construct a truth table for each of these compound propositions
\subitem a) $(p \land q) \to (p \lor q)$

\subitem
\begin{tabular}{| c | c | c | c | c |}
    \hline
	p & q & $p \land q$ & $ p \lor q$ & $(p \land q) \to (p \lor q)$ \\
	\hline
	T & T & T & T & T\\
	\hline
	T & F & F & T & T\\
	\hline
	F & T & F & T & T\\
	\hline
	F & F & F & F & T \\
	\hline
\end{tabular}
\subitem b) $(q \to \neg p) \leftrightarrow (p \leftrightarrow q)$
\subitem
\begin{tabular}{| c | c | c | c | c |}
	\hline
	p & q & $q \to \neg p$ & $p \leftrightarrow q$ &$(q \to \neg p) \leftrightarrow (p \leftrightarrow q)$\\
	\hline
	T & T & F & T & F\\
	\hline
	T & F & T & F & F\\
	\hline 
	F & T & T & F & F\\
	\hline
	F & F & T & T & T\\
	\hline
\end{tabular}
\subitem c) $(p \leftrightarrow q) \oplus (p \leftrightarrow \neg q) $
\subitem
\begin{tabular}{| c | c | c | c | c |}
	\hline
	p & q & $p \leftrightarrow q$ & $p \leftrightarrow \neg q$ & $(p \leftrightarrow q) \oplus (p \leftrightarrow \neg q) $\\
	\hline
	T & T & T & F & T \\
	\hline
	T & F & F & T & T \\
	\hline 
	F & T & F & T & T \\
	\hline 
	F & F & T & F & T \\
	\hline
\end{tabular}
\\
\item .
\\ Construct a truth table for each of these compound propositions
\subitem a) $(p \oplus q) \lor (p \oplus \neg q)$
\subitem
\begin{tabular}{| c | c | c | c | c |}
	\hline 
	p & q & $p \oplus q$ & $p \oplus \neg q$ & $(p \oplus q) \lor (p \oplus \neg q)$\\
    \hline
	T & T & F & T & T \\
	\hline
	T & F & T & F & T \\
	\hline 
	F & T & T & F & T \\
	\hline 
	F & F & F & T & T \\
	\hline
\end{tabular}
\subitem b) $(p \oplus q) \land (p \oplus \neg q)$
\subitem 
\begin{tabular}{| c | c | c | c | c |}
	\hline 
	p & q & $p \oplus q$ & $p \oplus \neg q$ &$(p \oplus q) \land (p \oplus \neg q)$\\
    \hline
	T & T & F & T & F \\
	\hline
	T & F & T & F & F \\
	\hline 
	F & T & T & F & F \\
	\hline 
	F & F & F & T & F \\
	\hline
\end{tabular}
\\
\\
\\
\item .\\
Is the assertion ``This statement is false`` a proposition?
\subitem According to the textbook, a proposition is a declarative sentence (that is, a sentence that declares a fact) that is either true
or false, but not both. The assertion ``This statement is false``  is neither true nor false. It is a paradox. So it is not a proposition 
\item .\\
A contestant in a TV game show is presented with three boxes, A, B, and C. He is told  that one of the boxes contains a prize, while the two others are empty. Each box has a statement written on it:\\
Box A: The prize is in this box\\
Box B: The prize is not in box A\\
Box C: The prize is not in this box\\
The host of the show tells the contestant that only one of the statements is true. Can the contestant find logically which box contains the prize? Justify your answer.\\

\subitem p: the prize is in box A
\subitem q: the prize is in box B
\subitem r: the prize is in box C
\subitem so the statement of box A is p; the statement of box B is $\neg p$; the statement of box C is $\neg r$
\subitem $\because$ there are only one price
\subitem $\therefore p \land q \equiv F$
\subitem $\therefore$ the truth table is:
\subitem
\begin{tabular}{| c | c | c | c | c | }
	\hline 
	p & r & p & $\neg p$ & $ \neg r$\\
	\hline
	T & F & T & F & T \\
	\hline 
	F & T & F & T & F \\
	\hline 
	F & F & F & T & T \\
	\hline
\end{tabular}
\subitem
\subitem According to the truth table, we can know that only when p is false and the r is true, there is only one true statement on the box. So the price is in box C.
\item .\\
This exercise relate to the inhabitants of the island of knights and knaves created by Smullyan, where knights always tell the truth and knaves always lie. You encounter two people, A and B. Determine, if possible, what A and B are if they address you in the way described. If you cannot determine what these two people are, can you draw any conclusions?\\
\\
A says “The two of us are both knights”, and B says “A is a knave”.\\
\subitem p: A is Knight, $\neg p$: A is knave
\subitem q: B is Knight, $\neg q$: B is knave
\subitem $\because$ knights always tell the truth and knaves always lie
\subitem $\therefore$ the value of the type of person, p, must equal to the value of his of her statement, r. 
\subitem According to the truth table below, we can know that the $\neg (p \oplus r)$ is true, if and only if when the knave lie or the knight tell truth.  
\subitem
\begin{tabular}{| c | c | c | }
	\hline 
	p & r & $\neg (p \oplus r)$\\
	\hline 
	T & T & T\\
	\hline
	T & F & F\\
	\hline 
	F & T & F\\
	\hline 
	F & F & T\\
	\hline
\end{tabular}
\subitem A : $p \land q$
\subitem B : $\neg p$
\subitem So we can know that :
\subitem $\neg (p \oplus (p \land q))$:\;A's statement is true when he is a knight, and is false when he is a knave
\subitem $ \neg (q \oplus \neg p)$:\;B's statement is true when he is a knight, and is false when he is a knave
\subitem
\begin{tabular}{| c | c | c | c | }
	\hline 
	p & q & $\neg (p \oplus (p \land q))$ & $ \neg (q \oplus \neg p)$\\
	\hline 
	T & T & T & F\\
	\hline
	T & F & F & T\\
	\hline 
	F & T & T & T\\
	\hline 
	F & F & T & F\\
	\hline
\end{tabular}
\subitem
\subitem According to the truth table, we can know that that A is a knave, and B is a knight. \\\\
\item .\\
Use truth tables to verify the associative laws:
\subitem a) $(p \lor q) \lor r  \Leftrightarrow p \lor (q \lor r) $
\subitem 
\begin{tabular}{| c | c | c | c | c | c | c |}
    \hline
	p & q & r & $p \lor q$ & $q \lor r$ & $(p \lor q) \lor r $ & $p \lor (q \lor r)$ \\
	\hline
	T & T & T & T & T & T & T\\
	\hline
	T & T & F & T & T & T & T\\
	\hline
	T & F & T & T & T & T & T\\
	\hline
	T & F & F & T & F & T & T\\
	\hline
	F & T & T & T & T & T & T\\
	\hline
	F & T & F & T & T & T & T\\
	\hline
	F & F & T & F & T & T & T\\
	\hline
	F & F & F & F & F & F & F\\
	\hline
\end{tabular}
\\
\subitem b) $(p \land q) \land r  \Leftrightarrow p \land (q \land r) $
\subitem 
\begin{tabular}{| c | c | c | c | c | c | c |}
    \hline
	p & q & r & $p \land q$ & $ q \land r$ &$(p \land q) \land r $ & $p \land (q \land r) $ \\
	\hline
	T & T & T & T & T & T & T\\
	\hline
	T & T & F & T & F & F & F\\
	\hline
	T & F & T & F & F & F & F\\
	\hline
	T & F & F & F & F & F & F\\
	\hline
	F & T & T & F & T & F & F\\
	\hline
	F & T & F & F & F & F & F\\
	\hline
	F & F & T & F & F & F & F\\
	\hline
	F & F & F & F & F & F & F\\
	\hline
\end{tabular}
\item .\\
Show that $p \leftrightarrow q$ and $(p \land q) \lor (\neg p \land \neg q)$are equivalent.
\subitem $LHS = p \leftrightarrow q $
\subitem $\qquad\quad = (p \to q) \land (q \to p)$
\subitem $\qquad\quad = (\neg p \land q) \land (\neg q \land p)$  
\subitem $\qquad\quad = (\neg p \lor q) \land (\neg q \lor p)$
\subitem $\qquad\quad = ((\neg p \lor q) \land \neg q)) \lor ((\neg p \lor q) \land p)$
\subitem $\qquad\quad = ((\neg p \land \neg q) \lor (q \land \neg q)) \lor (\neg p \land  p ) \lor (q \land q))$
\subitem $\qquad\quad = (\neg p \land \neg q) \lor (q \land q) $
\subitem $\qquad\quad = (p \land q) \lor (\neg p \land \neg q)$ 
\subitem $RHS = (p \land q) \lor (\neg p \land \neg q)$
\subitem $\therefore p \leftrightarrow q$ and $(p \land q) \lor (\neg p \land \neg q)$are equivalent.
\\
\item Extra Credit\\We are back on the island of knights and knaves (see exercise 5 above). John and Bill are residents.\\
John: if Bill is a knave, then I am a knight\\
Bill: we are different\\
Who is who?\\
\subitem p: Bill is a knight.
\subitem q: John is a knight.
\subitem $\therefore$ Bill's statement is $p \oplus q$
\subitem $\quad$ John's statement is $\neg p \to q$
\subitem
\begin{tabular}{| c | c | c | c | }
	\hline 
	p & q & $\neg (p \oplus (p \oplus q))$ & $ \neg (q \oplus (\neg p \to q))$\\
	\hline 
	T & T & F & T\\
	\hline
	T & F & T & F\\
	\hline 
	F & T & F & T\\
	\hline 
	F & F & T & T\\
	\hline
\end{tabular}
\subitem According to the truth table, we can know that both of them are knave.
\end{enumerate}
\end{document}