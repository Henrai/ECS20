
% Also check out Professor Matloff's guide: 
% http://heather.cs.ucdavis.edu/~matloff/LaTeX/HowToCreate.html.
\documentclass{article}

% Packages Used
\usepackage{fancyhdr} % Required for custom headers
\usepackage{lastpage} % Required to determine the last page for the footer
\usepackage{extramarks} % Required for headers and footers
\usepackage{graphicx} % Required to insert images
\usepackage{lipsum} % Used for inserting dummy 'Lorem ipsum' text into the template
\usepackage{comment}  % Used for multi-line commenting
\usepackage{booktabs} % For better looking tables
\usepackage{array}       % for better arrays (eg matrices) in maths
\usepackage{paralist}    % very flexible & customisable lists (eg. enumerate/itemize, etc.)
\usepackage{verbatim}  % adds environment for commenting out blocks of text & for better verbatim
\usepackage{subfig}      % make it possible to include more than one captioned figure/table in a single float
\usepackage{amsthm}   % make proofs look better
\usepackage{amsfonts}
\usepackage{amsmath}
\usepackage{amssymb}
\usepackage{eufrak}      % for fraktur fonts
\usepackage{mathabx}  % for \divides
\usepackage{enumerate} % to get lists enumerated with letters
\usepackage{hyperref}  % to get attractive URLs
\usepackage{bussproofs} % for setting proofs
\usepackage{etoolbox}
\usepackage{enumitem}
%algorithm
\usepackage{algorithmicx}
\usepackage[ruled]{algorithm}
\usepackage{algpseudocode}
\usepackage{algpascal}
\usepackage{algc}
%\algdisablelines
\newcommand{\alg}{\texttt{algorithmicx}}
\newcommand{\old}{\texttt{algorithmic}}
\newcommand{\euk}{Euclid}
\newcommand\ASTART{\bigskip\noindent\begin{minipage}[b]{0.5\linewidth}}
\newcommand\ACONTINUE{\end{minipage}\begin{minipage}[b]{0.5\linewidth}}
\newcommand\AENDSKIP{\end{minipage}\bigskip}
\newcommand\AEND{\end{minipage}}


% For theorem enviornment
\theoremstyle{definition}
\newtheorem{definition}{Definition}
\newtheorem{theorem}{Theorem}[section]
\newtheorem{corollary}{Corollary}[theorem]
\newtheorem{lemma}{Lemma}

\newtheorem{mathrule}{Rule}
\newtheorem{case}{Case}
\newtheorem{subcase}{Case}[case]

\theoremstyle{plain}
\newtheorem{example}{Example}
\newtheorem{problem}{Problem}[section]
\providecommand{\ceil}[1]{\left \lceil #1 \right \rceil }
\providecommand{\floor}[1]{\left \lfloor #1 \right \rfloor }
% For improved end of proof formatting
\patchcmd{\endproof}  % <cmd>
  {\endtrivlist}               % <search>
  {\endtrivlist\par\nobreak\vspace*{\dimexpr-\baselineskip-\parskip}\nobreak\noindent\hrulefill}% <replace>
  {}{}                            % <succes><failure>

% Margins
\topmargin=-0.45in
\evensidemargin=0in
\oddsidemargin=0in
\textwidth=6.5in
\textheight=9.0in
\headsep=0.25in 

\linespread{1.1} % Line spacing

% Set up the header and footer
\pagestyle{fancy}
\lhead{ECS20: Discrete Mathematics\\ UC Davis - Patrice Koehl} % Top left header
\chead{} % Top center header
\rhead{\firstxmark Anze Wang ID: 912777492\\ECS 020 A04} % Top right header
\lfoot{\lastxmark} % Bottom left footer
\cfoot{} % Bottom center footer
\rfoot{Page\ \thepage\ of\ \pageref{LastPage}} % Bottom right footer

\setlength\parindent{10pt} % Removes all indentation from paragraphs

% Common boolean operators.
%\newcommand*\AND{\wedge}
%\newcommand*\OR{\vee}
%\newcommand*\NOT{\neg}
%\newcommand*\IMPLIES{\implies}
%\newcommand*\XOR{\mathbin{\oplus}}


\begin{document}

\begin{center} \bf \LARGE Homework 6\\
\end{center}


\begin {enumerate}[itemindent=30pt,label=\bf Exercise {\arabic*}:]
\item .\\Prove or disprove each of these statements about the floor and ceiling functions.
\subitem a) $\ceil{\floor{x}} = \floor{x}$, for all real number x.
\begin{align*}
	&\text{let } x = n + \varepsilon, n \in \mathbb{Z}, \varepsilon \in \mathbb{R},  0 \leq \varepsilon < 1\\
	&LHS = \ceil{\floor{x}} = \ceil{n} = n\\
	&RHS = \floor{x} = n\\
	\therefore &LHS = RHS\\
	\text{So}&\text{ this statement is true}
\end{align*}
\subitem b) $\floor{xy} = \floor{x} \floor{y}$, for all real number x and y.
\subitem $\text{let } x = n_{x} + \varepsilon_{x}, n_{x} \in \mathbb{Z}, \varepsilon_{x} \in \mathbb{R},  0 \leq \varepsilon < 1$ , $\text{let } y = n_{y} + \varepsilon_{y}, n_{y} \in \mathbb{Z}, \varepsilon_{y} \in \mathbb{R},  0 \leq \varepsilon < 1$
\begin{align*}
	LHS &= \floor{n_{x}n_{y} + n_{x}\varepsilon_{y} + n_{y}\varepsilon_{x} + \varepsilon_{x} \varepsilon_{y}}  \\
	&= \floor{n_{x}n_{y} + n_{x}\varepsilon_{y} + n_{y}\varepsilon_{x}}\\
	RHS &= n_{x}n_{y}\\
	 \therefore LHS &\neq RHS\\
	 \text{So this}&\text{ statement is false}
\end{align*}

\subitem c) $\floor{\sqrt{\ceil{x}}} = \floor{\sqrt{x}}$, for all real number x.
\subitem let $x= 0.5$, then:
\begin{align*}
	LHS &= \floor{\sqrt{\ceil{x}}} = 1 \\
	RHS &= \floor{\sqrt{x}} = 0 \\
	LHS &\neq RHS \\
	\text{So this}&\text{ statement is false}
\end{align*}
\item .\\
Show that $x^{3}$ is $O(x^{4})$, but that $x^{4}$ is not $O(x^3)$.
\begin{align*}
	\mathrm{if }\;&x^{3} \leqslant c x^{4} \\
				& 1 \leqslant cx
\end{align*}
\begin{center}
This statement is true for fall $x >0$ and  $c>0$, so $x^{3} = O(x^{4})$
\end{center}
\begin{align*}
	\mathrm{if } \;&x^{4} \leqslant c x^{3} \\
				& x \leqslant c
\end{align*}
\subitem \qquad we cannot find an $x_{0}$ which make  $\forall x > x_{0}$, such that $0 < x < c$, So $x^{4} \neq O(x^{3})$
\newpage
\item .\\
\subitem a) Show that $2x - 9$ is $\Theta (x)$
\subitem \qquad From the definition of big Theta:
	$$c_1 x \leq 2x - 9 < c_2 x$$
\subitem \qquad For all $x \geq x_{0}$:
	$$c_1 \leq 2 - \dfrac{9}{x} \leq c_2$$
\subitem \qquad The right-hand inequality can be made to hold for any value if $x \geq 3$ by choosing $c_{2} \geq 1$
\subitem \qquad The left-hand inequality can be made to hold for any value if $x \geq 1$ by choosing $c_{1} \leq -7$
\subitem \qquad Thus, by choosing $c_1 = -7$, $c_2 = 1$, and $x_{0} = 3$, we can verify that $2x - 9 = \Theta (x)$\\
\subitem b) Show that $3x^2 + x - 5$ is $\Theta (x^2)$
\subitem \qquad From the definition of big Theta:
	$$c_1 x^2 \leq 3x^2 + x - 5 < c_2 x^2$$
\subitem \qquad For all $x \geq x_{0}$:
	$$c_1 \leq 3 + \dfrac{1}{x} - \dfrac{5}{x^2} \leq c_2$$
\subitem \qquad The right-hand inequality can be made to hold for any value if $x \geq 5$ by choosing $c_{2} \geq 4$
\subitem \qquad The left-hand inequality can be made to hold for any value if $x \geq 5$ by choosing $c_{1} \leq 3$
\subitem \qquad Thus, by choosing $c_1 = 3$, $c_2 = 4$, and $x_{0} = 5$, we can verify that $3x^2 + x - 5= \Theta (x^2)$\\
\subitem c) Show that $\floor{x + \dfrac{2}{3}}$ is $\Theta (x)$
\subitem \qquad From the definition of big Theta:
	$$c_1 x \leq \floor{x + \dfrac{2}{3}} < c_2 x$$
\subitem \qquad The right-hand inequality can be made to hold for any value if $x \geq 2$ by choosing $c_{2} \geq 3$
\subitem \qquad The left-hand inequality can be made to hold for any value if $x \geq 0$ by choosing $c_{1} \leq 1$
\subitem \qquad Thus, by choosing $c_1 = 1$, $c_2 = 3$, and $x_{0} = 2$, we can verify that $\floor{x + \dfrac{2}{3}}=\Theta (x)$
\subitem d) Show that $log_{10}(x) $ is $\Theta (log_{2} (x))$
\subitem \qquad From the definition of big Theta:
	$$c_1 log_{2}(x) \leq log_{10}(x) < c_2 log_{2}(x)$$
\subitem \qquad For all $x \geq x_{0}$:
	$$c_1 \leq \dfrac{1}{log_{2}10}  \leq c_2$$
\subitem \qquad The right-hand inequality can be made to hold for any value if $x \geq 0$ by choosing $c_{2} \geq \dfrac{1}{log_{2}10}$
\subitem \qquad The left-hand inequality can be made to hold for any value if $x \geq 5$ by choosing $c_{1} \leq \dfrac{1}{log_{2}10}$
\subitem \qquad Thus, by choosing $c_1 = \dfrac{1}{log_{2}10}$, $c_2 = \dfrac{1}{log_{2}10}$, and $x_{0} = 0$, we can verify that $\Theta (log_{2} (x))$\\
\item .\\
Describe an algorithm that uses only assignment statements that replaces the quadruplet(w,x,y,z) with (x,y,z,w). What is the minimum number of assignment statements needed?
\subitem the minimum number of assignments is 5.
\alglanguage{pseudocode}
\begin{algorithm}[h]
\caption{replaces the quadruplet }
\begin{algorithmic}[1]
\Procedure {replace}{$w,x,y,z$} \Comment {A is a sequence with n elements}
  \State $temp \gets w$
  \State $w \gets x$
  \State $x \gets y$
  \State $y \gets z$ 
  \State $z \gets temp$
\EndProcedure
\end{algorithmic}
\end{algorithm}

\item .\\Devise an algorithm for finding both the largest and the smallest integers in a finite sequence of integers.What is the complexity of your algorithm?
\subitem The complexity of the algorithm is $O(n)$.
\alglanguage{pseudocode}
\begin{algorithm}[h]
\caption{Find Max and Min}
\begin{algorithmic}[1]
\Procedure {getMaxAndMin}{$A$, $n$} \Comment {A is a sequence with n elements}
   \State $max \leftarrow -\infty$ \Comment {initialize max as negative infinity}
   \State $min \leftarrow \infty$ \Comment {initialize min as positive infinity}
   \For {$i \leftarrow 0, n-1$} \Comment {pick the $ith$ element in A}
		\If {$min > A(i)$}
			\State $min \gets A(i)$	\Comment{if $A(i) < min$, let A(i) become the new minimum}
		\EndIf
		\If {$max < A(i)$}
			\State {$max \gets A(i) $} \Comment{if $A(i) > max$, let A(i) become the new maximum}
		\EndIf
   \EndFor
   \State \textbf{return} $(min, max)$
\EndProcedure
\end{algorithmic}
\end{algorithm}
\newpage
\item Extra Credit\\We call a positive integer perfect if it equals the sum of its positive divisors other than itself.
\subitem a) Show that 6 and 28 are perfect
\begin{align*}
	&6 = 1*2*3	\\
	\therefore\; &\text{sum} = 1+2+3 = 6\\
	\therefore\; &\text{6 is a perfect number}\\ 
	&28 = 1*2*2*7 \\
	\therefore\; &\text{sum} = 1+2+7 + 4 + 14  = 28\\
	\therefore\; &\text{28 is a perfect number}\\
\end{align*} 
\subitem b) Show that $2^{p-1}(2^{p}-1)$ is a perfect number when $2p-1$ is prime.
\begin{align*}
	\text{sum} &= 2^{p-1} + (1 + 2^p -1)\sum \limits_{n = 0}^{p-2} 2^{n}\\
			   &= 2^{p-1} +  2^p\;(2^{p-2}-1) \\
			   &= 2^{p-1} +  2^{p-1}\;(2^{p-1}-2) \\
			   &= 2^{p-1}(2^{p}-1)
\end{align*}
\begin{center}
	so $2^{p-1}(2^{p}-1)$ is a perfect number
\end{center}
\end{enumerate}
\end{document}
