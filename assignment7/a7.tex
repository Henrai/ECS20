% Also check out Professor Matloff's guide: 
% http://heather.cs.ucdavis.edu/~matloff/LaTeX/HowToCreate.html.
\documentclass{article}

% Packages Used
\usepackage{fancyhdr} % Required for custom headers
\usepackage{lastpage} % Required to determine the last page for the footer
\usepackage{extramarks} % Required for headers and footers
\usepackage{graphicx} % Required to insert images
\usepackage{lipsum} % Used for inserting dummy 'Lorem ipsum' text into the template
\usepackage{comment}  % Used for multi-line commenting
\usepackage{booktabs} % For better looking tables
\usepackage{array}       % for better arrays (eg matrices) in maths
\usepackage{paralist}    % very flexible & customisable lists (eg. enumerate/itemize, etc.)
\usepackage{verbatim}  % adds environment for commenting out blocks of text & for better verbatim
\usepackage{subfig}      % make it possible to include more than one captioned figure/table in a single float
\usepackage{amsthm}   % make proofs look better
\usepackage{amsfonts}
\usepackage{amsmath}
\usepackage{amssymb}
\usepackage{eufrak}      % for fraktur fonts
\usepackage{mathabx}  % for \divides
\usepackage{enumerate} % to get lists enumerated with letters
\usepackage{hyperref}  % to get attractive URLs
\usepackage{bussproofs} % for setting proofs
\usepackage{etoolbox}
\usepackage{enumitem}
\usepackage{tabularx}
%algorithm
\usepackage{algorithmicx}
\usepackage[ruled]{algorithm}
\usepackage{algpseudocode}
\usepackage{algpascal}
\usepackage{algc}
%\algdisablelines
\newcommand{\alg}{\texttt{algorithmicx}}
\newcommand{\old}{\texttt{algorithmic}}
\newcommand{\euk}{Euclid}
\newcommand\ASTART{\bigskip\noindent\begin{minipage}[b]{0.5\linewidth}}
\newcommand\ACONTINUE{\end{minipage}\begin{minipage}[b]{0.5\linewidth}}
\newcommand\AENDSKIP{\end{minipage}\bigskip}
\newcommand\AEND{\end{minipage}}


% For theorem enviornment
\theoremstyle{definition}
\newtheorem{definition}{Definition}
\newtheorem{theorem}{Theorem}[section]
\newtheorem{corollary}{Corollary}[theorem]
\newtheorem{lemma}{Lemma}

\newtheorem{mathrule}{Rule}
\newtheorem{case}{Case}
\newtheorem{subcase}{Case}[case]

\theoremstyle{plain}
\newtheorem{example}{Example}
\newtheorem{problem}{Problem}[section]
\providecommand{\ceil}[1]{\left \lceil #1 \right \rceil }
\providecommand{\floor}[1]{\left \lfloor #1 \right \rfloor }
% For improved end of proof formatting
\patchcmd{\endproof}  % <cmd>
  {\endtrivlist}               % <search>
  {\endtrivlist\par\nobreak\vspace*{\dimexpr-\baselineskip-\parskip}\nobreak\noindent\hrulefill}% <replace>
  {}{}                            % <succes><failure>

% Margins
\topmargin=-0.45in
\evensidemargin=0in
\oddsidemargin=0in
\textwidth=6.5in
\textheight=9.0in
\headsep=0.25in 

\linespread{1.1} % Line spacing

% Set up the header and footer
\pagestyle{fancy}
\lhead{ECS20: Discrete Mathematics\\ UC Davis - Patrice Koehl} % Top left header
\chead{} % Top center header
\rhead{\firstxmark Anze Wang ID: 912777492\\ECS 020 A04} % Top right header
\lfoot{\lastxmark} % Bottom left footer
\cfoot{} % Bottom center footer
\rfoot{Page\ \thepage\ of\ \pageref{LastPage}} % Bottom right footer

\setlength\parindent{10pt} % Removes all indentation from paragraphs

% Common boolean operators.
%\newcommand*\AND{\wedge}
%\newcommand*\OR{\vee}
%\newcommand*\NOT{\neg}
%\newcommand*\IMPLIES{\implies}
%\newcommand*\XOR{\mathbin{\oplus}}


\begin{document}

\begin{center} \bf \LARGE Homework 7\\
\end{center}


\begin {enumerate}[itemindent=30pt,label=\bf Exercise {\arabic*}:]
\item .\\
\subitem e)-2002 is divided by 89?
\subitem \qquad $\because -2002\;mod\;89 = 45$
\subitem \qquad $\therefore$ -2002 is not divided by 89
\subitem f) 0 is divided by 19?
\subitem \qquad $\because 0\;mod\;19 = 0$
\subitem \qquad $\therefore$ 0 is divided by 19
\subitem g) 1,234,567 is divided by 101?
\subitem \qquad $\because 1234567\;mod\;19 = 14$
\subitem \qquad $\therefore$ 1,234,567 is not divided by 19
\subitem h)-100 is divided by 103?
\subitem \qquad $\because -100\;mod\;103 = 3$
\subitem \qquad $\therefore$ -100 is not divided by 103
\item .\\
\subitem a) Let a be a positive integer. Show that $gcd(a,a-1) = 1$
\begin{align*}
	 &gcd(a, a-1) \\
    =&gcd(a-1, 1) \\
    =&1
\end{align*}
\subitem b) Use the result of part a) to solve the Diophantin eequation $$a+2b=2ab$$\qquad \qquad  where $(a,b)$ are positive integers
\begin{align*}
	& a+2b = 2ab\\
	& a = 2(a-1)b\\
	& b = \dfrac{a}{2(a-1)}
\end{align*}
\subitem \qquad Because both a and b are positive integer, we can only get one set of solution, a = 2, b = 1. 
\item Extra Credit\\ Let a and b be two strictly positive integers. Solve $gcd(a,b)+lcm(a,b) = b + 9$
\subitem Let $a = pd$, $b = qd$ ,which p, q, d are positive integers
\begin{align*}
	& gcd(a,b)+lcm(a,b) = b + 9 \\
	& pd + \dfrac{pqd^2}{d} = qd + 9\\
	& pd + pqd - qd = 9\\
	& d(p - q + pq) = 9\\
\end{align*}
\subitem Because p ,q ,d are positive integers, we can get these solutions:\\
\begin{tabularx}{300pt}{XXX} 
\qquad &
\begin{equation*}
  \left\{
   \begin{aligned}
   &d = 9  \\
   &p = 1  \\
   &q = 1  \\
   \end{aligned}
  \right.
\end{equation*} &
  \begin{equation*}
  \left\{
   \begin{aligned}
   	d = 3\\
    p = 2\\
    q = 1\\
   \end{aligned}
  \right.
\end{equation*}
\end{tabularx}
\subitem So we can know that $a = 9, b = 9$ and $a = 6 , b = 3$
\end{enumerate}
\end{document}
