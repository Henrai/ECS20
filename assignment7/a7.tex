% Also check out Professor Matloff's guide: 
% http://heather.cs.ucdavis.edu/~matloff/LaTeX/HowToCreate.html.
\documentclass{article}

% Packages Used
\usepackage{fancyhdr} % Required for custom headers
\usepackage{lastpage} % Required to determine the last page for the footer
\usepackage{extramarks} % Required for headers and footers
\usepackage{graphicx} % Required to insert images
\usepackage{lipsum} % Used for inserting dummy 'Lorem ipsum' text into the template
\usepackage{comment}  % Used for multi-line commenting
\usepackage{booktabs} % For better looking tables
\usepackage{array}       % for better arrays (eg matrices) in maths
\usepackage{paralist}    % very flexible & customisable lists (eg. enumerate/itemize, etc.)
\usepackage{verbatim}  % adds environment for commenting out blocks of text & for better verbatim
\usepackage{subfig}      % make it possible to include more than one captioned figure/table in a single float
\usepackage{amsthm}   % make proofs look better
\usepackage{amsfonts}
\usepackage{amsmath}
\usepackage{amssymb}
\usepackage{eufrak}      % for fraktur fonts
\usepackage{mathabx}  % for \divides
\usepackage{enumerate} % to get lists enumerated with letters
\usepackage{hyperref}  % to get attractive URLs
\usepackage{bussproofs} % for setting proofs
\usepackage{etoolbox}
\usepackage{enumitem}
\usepackage{tabularx}
%algorithm
\usepackage{algorithmicx}
\usepackage[ruled]{algorithm}
\usepackage{algpseudocode}
\usepackage{algpascal}
\usepackage{algc}
%\algdisablelines
\newcommand{\alg}{\texttt{algorithmicx}}
\newcommand{\old}{\texttt{algorithmic}}
\newcommand{\euk}{Euclid}
\newcommand\ASTART{\bigskip\noindent\begin{minipage}[b]{0.5\linewidth}}
\newcommand\ACONTINUE{\end{minipage}\begin{minipage}[b]{0.5\linewidth}}
\newcommand\AENDSKIP{\end{minipage}\bigskip}
\newcommand\AEND{\end{minipage}}


% For theorem enviornment
\theoremstyle{definition}
\newtheorem{definition}{Definition}
\newtheorem{theorem}{Theorem}[section]
\newtheorem{corollary}{Corollary}[theorem]
\newtheorem{lemma}{Lemma}

\newtheorem{mathrule}{Rule}
\newtheorem{case}{Case}
\newtheorem{subcase}{Case}[case]

\theoremstyle{plain}
\newtheorem{example}{Example}
\newtheorem{problem}{Problem}[section]
\providecommand{\ceil}[1]{\left \lceil #1 \right \rceil }
\providecommand{\floor}[1]{\left \lfloor #1 \right \rfloor }
% For improved end of proof formatting
\patchcmd{\endproof}  % <cmd>
  {\endtrivlist}               % <search>
  {\endtrivlist\par\nobreak\vspace*{\dimexpr-\baselineskip-\parskip}\nobreak\noindent\hrulefill}% <replace>
  {}{}                            % <succes><failure>

% Margins
\topmargin=-0.45in
\evensidemargin=0in
\oddsidemargin=0in
\textwidth=6.5in
\textheight=9.0in
\headsep=0.25in 

\linespread{1.1} % Line spacing

% Set up the header and footer
\pagestyle{fancy}
\lhead{ECS20: Discrete Mathematics\\ UC Davis - Patrice Koehl} % Top left header
\chead{} % Top center header
\rhead{\firstxmark Anze Wang ID: 912777492\\ECS 020 A04} % Top right header
\lfoot{\lastxmark} % Bottom left footer
\cfoot{} % Bottom center footer
\rfoot{Page\ \thepage\ of\ \pageref{LastPage}} % Bottom right footer

\setlength\parindent{10pt} % Removes all indentation from paragraphs

% Common boolean operators.
%\newcommand*\AND{\wedge}
%\newcommand*\OR{\vee}
%\newcommand*\NOT{\neg}
%\newcommand*\IMPLIES{\implies}
%\newcommand*\XOR{\mathbin{\oplus}}


\begin{document}

\begin{center} \bf \LARGE Homework 7\\
\end{center}


\begin {enumerate}[itemindent=30pt,label=\bf Exercise {\arabic*}:]
\item .\\
\subitem e)-2002 is divided by 89?
\subitem \qquad $\because -2002\;mod\;89 = 45$
\subitem \qquad $\therefore$ -2002 is not divided by 89
\subitem f) 0 is divided by 19?
\subitem \qquad $\because 0\;mod\;19 = 0$
\subitem \qquad $\therefore$ 0 is divided by 19
\subitem g) 1,234,567 is divided by 101?
\subitem \qquad $\because 1234567\;mod\;19 = 14$
\subitem \qquad $\therefore$ 1,234,567 is not divided by 19
\subitem h)-100 is divided by 103?
\subitem \qquad $\because -100\;mod\;103 = 3$
\subitem \qquad $\therefore$ -100 is not divided by 103
\item .\\
\subitem a) Let a be a positive integer. Show that $gcd(a,a-1) = 1$
\begin{align*}
	 &gcd(a, a-1) \\
    =&gcd(a-1, 1) \\
    =&1
\end{align*}
\subitem b) Use the result of part a) to solve the Diophantin eequation $$a+2b=2ab$$\qquad \qquad  where $(a,b)$ are positive integers
\begin{align*}
	& a+2b = 2ab\\
	& a = 2(a-1)b\\
	& b = \dfrac{a}{2(a-1)}
\end{align*}

\subitem \qquad Because both a and b are positive integer, we can only get one set of solution, a = 2, b = 1. 
\newpage
\item .\\Leta, b, and c be three integers. Show that the equation $ax + by = c$ has at least one solution $(x_1,y_1)$ if and only if $gcd(a,b) \mid c$ \\
\subitem Let p be the proposition ``$ax + by = c$ has at least one integer solution $(x_1,y_1)$`` and q be the proposition ``$gcd(a,b) / c$``. We want to show that $p \leftrightarrow q$, which is logically equivalent to show that $p \leftarrow q$ and $q \rightarrow p$.
\subitem i) Let us show $p \rightarrow q$: 
\subitem Hypothesis: p is true, If $ax + by = c$ has one integer solution $(x_1,y_1)$, then let $a = pd$, $b = qd$ where $d = gcd(a,b)$ and p,q are also integers. We get:
\begin{align*}
	&ax + by = c\\
	&pdx + qdy = c\\
	&d(px + qy) = c
\end{align*}
\subitem \qquad Because $px + qy$ and c are integers, we can get that $gcd(a,b) \mid c$
\subitem ii) Let us show $q \rightarrow p$:
\subitem Hypothesis: q is true. If $gcd(a,b) \mid c$, then $c = k gcd(a,b)$, $k \in \mathbb{Z}$. And according to the BÉZOUT’S THEOREM, $gcd(a,b) = sa + tb $, $(a,b) \in \mathbb{Z}^2$ we can get that:
\begin{align*}
	c &= k\; gcd(a,b) \\
	  &= a \cdot sk + b \cdot tk\\	
\end{align*} 
\subitem  So we can get one solution easily:
\begin{equation*}
  \left\{
   \begin{aligned}
   	x = sk\\
   	y = tk
   \end{aligned}
  \right.
\end{equation*}
\subitem Thus this equation has at least one integer solution when $gcd(a,b) \mid c$
\subitem we can conclude that the equation $ax + by = c$ has at least one solution $(x_1,y_1)$ if and only if $gcd(a,b) \mid c$
\item .\\Let a, b and n be three positive integers with $gcd(a,n) = 1$ and $gcd(b,n) = 1$. Show that $gcd(ab,n) = 1$.
\subitem \qquad Because $gcd(a,n) = 1$, a and n are relative prime. \subitem \qquad So $a = pn + i$, $p \in \mathbb{R}$, $i \in \mathbb{N}, i < n$, i and n are relative prime.
\subitem \qquad Similarly, we can get $b = qn + j$, $q \in \mathbb{R}$ and $j \in \mathbb{N}, j < n$, i and n are relative prime.
\subitem \qquad Then $ab = pqn^{2} + pjn + qin + ij$
\subitem \qquad Thus:
\begin{align*}
	gcd(ab, n) &= gcd(n, ab\; mod \;n) \\
			   &= gcd(n, (pqn^{2} + pjn + qin + ij)\; mod \;n) \\
			   &= gcd(n,ij\; mod \;n)
\end{align*}
\subitem Because i, n are relative prime and j, n are relative prime, ij and n are relative prime. Therefore, $ij\;mod\;n$ and n are relative prime.
$$gcd(n, ij\; mod \;n) = gcd(ab, n) = 1$$

\item .\\ Prove that there are no solutions in integers x and y to the equation $3x^2+5y^2=19$.
\subitem Let both side mod 3:
\begin{align*}
	&RHS\;mod\;3\\
	=&19\;mod\;3\\
	=& 1 
\end{align*}
\begin{align*}
	&LHS\;mod\;3\\
	=&(3x^2+5y^2)\;mod\;3\\
	=& 5y^2\;mod\;3
\end{align*}
\subitem if $y = 1$, $5y^2\;mod\;3$ = 2
\subitem if $y = 2$  $5y^2 = 20 > 19$
\subitem So we can conclude that this equation cannot have integer solution.
\item .\\Show that if $n>3$ then $n$, $2n+1$ and $4n+1$ cannot all be prime
\subitem if $n \;mod \; 3 = 0$: n is not prime.
\subitem if $n \;mod\; 3 = 1 $, then $n = 3k + 1$, $k \in \mathbb{Z}$, then $2n + 1 = 6k + 3$, which is not prime.
\subitem if $n \;mod\; 3 = 2$, then $n = 3k + 2$, $k \in \mathbb{Z}$, then $4n + 1 = 12k + 9$, which is not prime.
\item .\\Prove or disprove that there are three consecutive odd positive integers that are primes,that is, odd primes of the form $p$, $p+2$, $p+4$.
\subitem Because $3$, $5$, $7$ are consecutive odd positive primes, this statement is true.
\item .\\Prove that if n is a positive integer such that the sum of its divisors is $n+1$, then n is prime.
\subitem let's assume that the sum of its divisors is $n+1$, and n is not a prime.
\subitem Because n is not a prime, $ n = 1*p*q$, $(p,q) \in \mathbb{Z}^{+}$
\subitem Because m,n may not be prime, the sum of n's divisors must greater than  or equal to $1 + p + q + n $, which is greater than $n +1$. This is contradict with my assumption. So we can conclude that if n is a positive integer such that the sum of its divisors is $n+1$, then n is prime.
\item Extra Credit\\ Let a and b be two strictly positive integers. Solve $gcd(a,b)+lcm(a,b) = b + 9$
\subitem Let $a = pd$, $b = qd$ ,which p, q, d are positive integers
\begin{align*}
	& gcd(a,b)+lcm(a,b) = b + 9 \\
	& d + \dfrac{pqd^2}{d} = qd + 9\\
	& d + pqd - qd = 9\\
	& d(1 - q + pq) = 9\\
\end{align*}
\subitem Because p ,q ,d are positive integers, we can get these solutions:\\
\begin{tabularx}{300pt}{XXXXXX} 
\qquad &
  \begin{equation*}
  \left\{
   \begin{aligned}
   	d = 3\\
    p = 3\\
    q = 1\\
   \end{aligned}
  \right.
\end{equation*}&
\begin{equation*}
  \left\{
   \begin{aligned}
   	d = 1\\
    p = 3\\
    q = 4\\
   \end{aligned}
  \right.
\end{equation*}&
\begin{equation*}
  \left\{
   \begin{aligned}
   	d = 1\\
    p = 9\\
    q = 1\\
   \end{aligned}
  \right.
\end{equation*} &
\begin{equation*}
  \left\{
   \begin{aligned}
   	d = 1\\
    p = 5\\
    q = 2\\
   \end{aligned}
  \right.
\end{equation*} &
\begin{equation*}
  \left\{
   \begin{aligned}
   &d = 9  \\
   &p = 1  \\
   &q = n, n \in \mathbb{Z}, n > 1  \\
   \end{aligned}
  \right.
\end{equation*} 
\end{tabularx}
\subitem So we can know that $(9, 3)$, $(3, 4)$,$(9, 1)$, $(5, 2)$ or $(9, 9n)$
\end{enumerate}
\end{document}
